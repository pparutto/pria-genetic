\section{Theory}

\begin{frame}{Motivations}
  \begin{block}{Algorithmic problems}
    \begin{itemize}
      \item Stochastic
      \item Convergence
    \end{itemize}
  \end{block}

  \begin{block}{Modelisation problems}
    \begin{itemize}
      \item How to encode a problem?
      \item Size of the initial population?
      \item Mutation operator?
      \item Recombination operator?
      \item Individual selection?
    \end{itemize}
  \end{block}
\end{frame}

\subsection{Schemata Theorem}
\begin{frame}{Presentation}
  \begin{block}{Goal}
    \begin{itemize}
    \item Defined by J.Holland in 1975\cite{holland1992}.
    \item Represent a group of indidual who all share some caracteristics.
    \item One schema, two schemata ;)
    \end{itemize}
  \end{block}    

  \begin{block}{Definition}
    \begin{itemize}
    \item Defined on the alphabet: $\{0;1;\#\}$
    \item $\#$ is "Don't care": it can be 0 or 1
    \end{itemize}
  \end{block}
\end{frame}

\begin{frame}{Example and mesures}
  \begin{block}{Example}
    \begin{itemize}
    \item The schema: $H = 1\textcolor{red}{\#}10\textcolor{red}{\#}1\textcolor{red}{\#}$
    \item Represents the following individuals:
      \begin{center}
        $1\textcolor{red}{0}10\textcolor{red}{0}1\textcolor{red}{0}$,
        $1\textcolor{red}{0}10\textcolor{red}{0}1\textcolor{red}{1}$,
        $1\textcolor{red}{0}10\textcolor{red}{1}1\textcolor{red}{0}$,
        $1\textcolor{red}{0}10\textcolor{red}{1}1\textcolor{red}{1}$,
        $1\textcolor{red}{1}10\textcolor{red}{0}1\textcolor{red}{0}$,
        $\ldots$
      \end{center}
    \end{itemize}
  \end{block}

  \begin{block}{Measures}
    \begin{itemize}
      \item Order(o): number of fixed positions
      \item Defining length($\delta$): distance between the first and last defined positions.
      \item Here: $o(H) = 4$ and $\delta(H) = 5$
      \item Intuitively: schemata with low o and short $\delta$ have an higher probability of surviving from a generation to the next.
    \end{itemize}
  \end{block}
\end{frame}

\begin{frame}{Possible evolutions : selection}
  \begin{block}[Hypthesis]
  \item Assume a fitness proportional reproduction
  \item ...
  \end{block}
\end{frame}

\begin{frame}{Result}
  du dada da
\end{frame}

\subsection{Extended schemata theorem}
\begin{frame}{Extended hypothesis}
  lol
\end{frame}
